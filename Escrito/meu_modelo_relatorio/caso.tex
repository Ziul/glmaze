
%Nessa seção deve ser apresentado pelo menos um exemplo de caso de teste. Se não for especificado na descrição do problema, ela deve definida, explicada e ilustrada pelos autores.

%Os exemplos de estudos de casos podem ser observados no apêndice \ref{cases}, onde estão apresentados os seguintes estudos de caso:
\rowcolors{1}{lightgray}{white}

Foram feitos três estudos de casos referentes ao programa em sí.

\subsection{Sistema de derrota}\label{label}

Para o primeiro caso é estudado a situação de derrota. É a situação onde o jogador colide com um fantasma.\\
	
\begin{tabular}{>{\centering}p{3.5cm}<{\centering}p{3.5cm}}
\hline
Pré-condições  & Ter iniciado o programa
\tabularnewline
\hline
Procedimentos 
& 1. Usando as teclas WSAD e o mouse, andar na direção de um inimigo.\tabularnewline
& 2. Colidir com o inimigo.
\tabularnewline\hline
Resultado Esperado & Musica de derrota é tocada. Jogador perde uma vida e retorna para a tela principal com uma mensagem de derrota.
\tabularnewline\hline
Pós-condições & Câmera do jogador parada olhando para o muro.
\end{tabular}

\subsection{Sistema de movimento}\label{mov}

Neste segundo estudo é verificado a condição primaria do programa, ou seja, iniciar o jogo e movimentar-se pelo cenário.\\

\begin{tabular}{>{\centering}p{3.5cm}<{\centering}p{3.5cm}}
\hline
Pré-condições  & Ter iniciado o programa
\tabularnewline
\hline
Procedimentos
& 1. Apertar sobre o botão quadrado no centro para iniciar o jogo.\tabularnewline 
& 2. Usar as teclas WSAD para se movimentar.\tabularnewline
& 3. Segurar o botão esquerdo do mouse e movimenta-lo para mover a direção da câmera.
\tabularnewline\hline
Resultado Esperado 
& A musica é alterada. Mostra a câmera do jogador e permite move-la com WSAD.  Permite mover a direção da câmera com o  mouse ao apertá-lo. 
\tabularnewline\hline
Pós-condições & É alterada a posição do jogador no ambiente. Ao fazer de uso do mouse, é alterado a direção de visão do jogador.
\end{tabular}

\subsection{Sistema de colisão}\label{label}

Neste terceiro estudo é verificado a condição de colisão com objetos. Para tal é verificado a colisão com a parede.\\
	
\begin{tabular}{>{\centering}p{3.5cm}<{\centering}p{3.5cm}}
\hline
Pré-condições  & Ter iniciado o programa
\tabularnewline
\hline
Procedimentos 
& 1. Usar o mouse para apontar a câmera para a direção de um muro.\tabularnewline
& 2. Usar a tecla W para seguir em frente e tentar atravessar o muro.
\tabularnewline\hline
Resultado esperado & O programa não deixo a câmera do jogador ultrapassar o muro e para o seu movimento.
\tabularnewline\hline
Pós-condições & Câmera do jogador parada olhando para o muro.
\end{tabular}


