\documentclass[journal,compsoc]{IEEEtran}\newcommand{\journal}{true}
%\documentclass[conference]{IEEEtran}\newcommand{\journal}{false}
\ifCLASSOPTIONcompsoc
  % IEEE Computer Society needs nocompress option
  % requires cite.sty v4.0 or later (November 2003)
  \usepackage[nocompress]{cite}
\else
  % normal IEEE
   \usepackage{cite}
\fi

%\providecommand{\tabularnewline}{\\}

\usepackage{indentfirst}

%%--------------------Colocados por minha vontade própria-----------------------
\usepackage{marvosym}
\usepackage[utf8x]{inputenc}
\usepackage[english,brazil]{babel}	
\usepackage[T1]{fontenc}
\usepackage{wasysym}				% setas
\usepackage{picinpar}				% figuras onde eu quero, problemas com referencias
\usepackage{sidecap}				% legenda ao lado da figura
\usepackage[small,bf]{caption}
\usepackage{lipsum}					% Pra encher linguiça
\usepackage{ifthen}

%\usepackage{fancyhdr}
%\lhead{O que quero no cabeçalho parte esquerda}
%\chead{O que quero no cabeçalho parte central}
%\rhead{O que quero no cabeçalho parte direita}
%\lfoot{O que quero no rodapé parte esquerda}
%\cfoot{O que quero no rodapé parte central}
%\rfoot{O que quero no rodapé parte direita}

\input nomes




%%------------------------------------------------------------------------------
%%	Resolveu o problema de a linguagem não estar declarada
\makeatletter
\def\markboth#1#2{\def\leftmark{\@IEEEcompsoconly{\sffamily}\MakeUppercase{\protect#1}}%
\def\rightmark{\@IEEEcompsoconly{\sffamily}\MakeUppercase{\protect#2}}}
\makeatother
%%------------------------------------------------------------------------------
%%----------------Para inclusão de algoritmos em C/C++/Assembly---------------------------
\usepackage{listings}
%\lstnewenvironment{C}
\lstset{%
language=C,							%linguagem
numbers=left,						%posição dos números
stepnumber=1,						%frequencia de aparição dos números
numbersep=3pt,
tabsize=4,
commentstyle=\color{blue},
basicstyle=\scriptsize\ttfamily}
% Inclua o source com: \lstinputlisting[language=C]{src/src.c}

\lstdefinelanguage{MSP430}
{ %% Se estiver faltando alguma palavra reservada, insira ela aqui
morekeywords={RRC,SWPB,RRA,SXT,PUSH,CALL,RETI,JNE,JNZ,JEQ,
Z,JNC,JLO,JC,JHS,JN,JGE,JL,JMP,NOP,POP,BR,
RET,CLRC,SETC,CLRZ,SETZ,CLRN,SETN,DINT,EINT,
RLA,RLC,ADD,ADDC,INV,XOR,CLR,MOV,TST,CMP,DEC,
SUB,SUBC,DECD,INC,INCD,SBC,MOV,DADD,BIT,
BIC,BIS,XOR,AND,MOV.B,MOV.W},
sensitive=false,
morecomment=[l]{//},
morecomment=[l]{;},
morecomment=[s]{/*}{*/},
}
% Inclua o source com: \lstinputlisting[language=MSP430]{src/src.asm}


%
%%------------------------------------------------------------------------------
%%---------------------Includes importantes para suporte------------------------
\input link							% Para verificar links de citações e outras configurações
%%------------------------------------------------------------------------------
%%------------------------------------------------------------------------------



% *** GRAPHICS RELATED PACKAGES ***
%

\ifCLASSINFOpdf
\usepackage{tikz}
\usetikzlibrary{shapes,arrows}
% \usepackage{tikzgraphicx}
%  \usepackage[pdftex]{graphicx}  
%  \pdfcompresslevel=1
  % declare the path(s) where your graphic files are
  % \graphicspath{{../pdf/}{../jpeg/}}
  % and their extensions so you won't have to specify these with
  % every instance of \includegraphics
  % \DeclareGraphicsExtensions{.pdf,.jpeg,.png}
\else
  \usepackage[dvips]{graphicx}
  % declare the path(s) where your graphic files are
  % \graphicspath{{../eps/}}
  % and their extensions so you won't have to specify these with
  % every instance of \includegraphics
  % \DeclareGraphicsExtensions{.eps}
\fi
\pdfcompresslevel=9



% *** ALIGNMENT PACKAGES ***
%
\usepackage{array}
% Frank Mittelbach's and David Carlisle's array.sty patches and improves
% the standard LaTeX2e array and tabular environments to provide better
% appearance and additional user controls. As the default LaTeX2e table
% generation code is lacking to the point of almost being broken with
% respect to the quality of the end results, all users are strongly
% advised to use an enhanced (at the very least that provided by array.sty)
% set of table tools. array.sty is already installed on most systems. The
% latest version and documentation can be obtained at:
% http://www.ctan.org/tex-archive/macros/latex/required/tools/


%\usepackage{mdwmath}
%\usepackage{mdwtab}
% Also highly recommended is Mark Wooding's extremely powerful MDW tools,
% especially mdwmath.sty and mdwtab.sty which are used to format equations
% and tables, respectively. The MDWtools set is already installed on most
% LaTeX systems. The lastest version and documentation is available at:
% http://www.ctan.org/tex-archive/macros/latex/contrib/mdwtools/


% IEEEtran contains the IEEEeqnarray family of commands that can be used to
% generate multiline equations as well as matrices, tables, etc., of high
% quality.


\usepackage{eqparbox}
% Also of notable interest is Scott Pakin's eqparbox package for creating
% (automatically sized) equal width boxes - aka "natural width parboxes".
% Available at:
% http://www.ctan.org/tex-archive/macros/latex/contrib/eqparbox/





% *** SUBFIGURE PACKAGES ***
%\ifCLASSOPTIONcompsoc
\usepackage[tight,normalsize,sf,SF]{subfigure}
%\else
%\usepackage[tight,footnotesize]{subfigure}
%\fi
% subfigure.sty was written by Steven Douglas Cochran. This package makes it
% easy to put subfigures in your figures. e.g., "Figure 1a and 1b". For IEEE
% work, it is a good idea to load it with the tight package option to reduce
% the amount of white space around the subfigures. Computer Society papers
% use a larger font and \sffamily font for their captions, hence the
% additional options needed under compsoc mode. subfigure.sty is already
% installed on most LaTeX systems. The latest version and documentation can
% be obtained at:
% http://www.ctan.org/tex-archive/obsolete/macros/latex/contrib/subfigure/
% subfigure.sty has been superceeded by subfig.sty.




% *** FLOAT PACKAGES ***
%
\usepackage{float}
\usepackage{fixltx2e}
% http://www.ctan.org/tex-archive/macros/latex/base/


%\usepackage{stfloats}
% stfloats.sty was written by Sigitas Tolusis. This package gives LaTeX2e
% the ability to do double column floats at the bottom of the page as well
% as the top. (e.g., "\begin{figure*}[!b]" is not normally possible in
% LaTeX2e). It also provides a command:
%\fnbelowfloat
% to enable the placement of footnotes below bottom floats (the standard
% LaTeX2e kernel puts them above bottom floats). This is an invasive package
% which rewrites many portions of the LaTeX2e float routines. It may not work
% with other packages that modify the LaTeX2e float routines. The latest
% version and documentation can be obtained at:
% http://www.ctan.org/tex-archive/macros/latex/contrib/sttools/
% Documentation is contained in the stfloats.sty comments as well as in the
% presfull.pdf file. Do not use the stfloats baselinefloat ability as IEEE
% does not allow \baselineskip to stretch. Authors submitting work to the
% IEEE should note that IEEE rarely uses double column equations and
% that authors should try to avoid such use. Do not be tempted to use the
% cuted.sty or midfloat.sty packages (also by Sigitas Tolusis) as IEEE does
% not format its papers in such ways.




% *** PDF, URL AND HYPERLINK PACKAGES ***
%
\usepackage{url}
% http://www.ctan.org/tex-archive/macros/latex/contrib/misc/
% 
% \url{my_url_here}.




% *** Do not adjust lengths that control margins, column widths, etc. ***
% *** Do not use packages that alter fonts (such as pslatex).         ***
% There should be no need to do such things with IEEEtran.cls V1.6 and later.
% (Unless specifically asked to do so by the journal or conference you plan
% to submit to, of course. )


% correct bad hyphenation here
\hyphenation{op-tical net-works semi-conduc-tor}

\begin{document}
	\title{%\includegraphics[width=18cm]{./fts/cap.png}\\
	\hell\\\ver}
	\author{
%--------------------------------Nomes------------------------------------------
%		\hyperref[luiz]{Luiz Fernando Gomes de Oliveira},~\IEEEmembership{10/46969} 
%		\\\hyperref[panda]{Helbert de Oliveira Coelho Junior},~\IEEEmembership{10/45253}
\ifthenelse{\equal{\journal}{true}}{
\allnames\\
}{
%\thanks{Revisado em \today.}
\IEEEauthorblockN{\luiz}
\IEEEauthorblockA{Matricula: \luizmatricula\\E-mail: \eluiz}
\and
\IEEEauthorblockN{\gust}
\IEEEauthorblockA{Matricula: \gustmatricula\\E-mail: \egust}
\and
\IEEEauthorblockN{\fay}
\IEEEauthorblockA{Matricula: \faymatricula\\E-mail: \efay}
}}

%--------------------------O titulo é inserido aqui!----------------------------
%	\selectlanguage{english}  %seleciona o idioma em inglês [para mater os termos abstract e o thanks em inglês]
	\markboth{Universidade de Bras\'ilia - Campus Gama - FGA, \entrega}%
%{	\begin{figure*}
%    	\includegraphics[width=\textwidth]{./fts/unb.jpg}
%	\end{figure*}
%}
	{Shell \MakeLowercase{\textit{et al.}}: \ver}
%-------------------------------------------------------------------------------

	\IEEEcompsoctitleabstractindextext{%
	\begin{abstract}
		% objetivo do experimento
% resumo do procedimento (forma como o experimento foi executado)
% os resultados sucintamente discutidos
% Deve ser feito de preferencia em Inglês
%
%
\selectlanguage{brazil}  %%Desta linha em diante coloque o resumo apenas em português. Nas linhas anteriores, mantenha o texto em inglês.
%
Abstract
%


	\end{abstract}}
	%%-------------------------------------
%	\begin{IEEEkeywords} %palavras chaves
%		\keyw %,\LaTeX.
%	\end{IEEEkeywords}}
	% make the title area
%\onecolumn		%Conferir se pode ser feito em duas colunas

	\maketitle

	\IEEEdisplaynotcompsoctitleabstractindextext
	\IEEEpeerreviewmaketitle

	\selectlanguage{brazil}		%voltando para o português.
	%\tableofcontents			% Sumario

	\section{Introdução}\label{intro}
		%Definir, caracterizar e mostrar um breve historico e as utilidades do foco



%Segue um exemplo de como INICIAR o texto. Da segunda palavra em
%diante, escreva normalmente
%	\IEEEPARstart{T}{his} demo file is intended to serve as a ``starter file''
%	for IEEE Computer Society journal papers produced under \LaTeX\ using
%	IEEEtran.cls version 1.7 and later.
%
%	\IEEEPARstart{D}{escrição}, com textos e imagens, da configuração do procedimento e cenário experimental. Explorar técnicas e métodos de medida 
%
\nocite{dsl} %Apresenta a referência, porém não cita no texto
\subsection{Objetivos}\label{obj}

\IEEEPARstart{E}{ste} experimento tem como objetivo descrever um passo a passo de como obter um ambiente básico para as demais aulas. Para isso, precisamos ter uma distribuição contendo os aplicativos fundamentais para o desenvolvimento da matéria de Sistemas Embarcados.
	
\subsection{Introdução Teórica}\label{intro}	
\subsubsection{Damn Small Linux}\label{dsl}
\textit{Damn Small Linux}, ou apenas DSL, é uma distribuição versátil do Linux, tendo o tamanho de cerca de 50mb e ocupando como espaço final na RAM cerca de 128MB. A distribuição foi montada com o intuito de cumprir os seguintes objetivos:
	
%\hfill \today

\subsubsection{Material}\label{mat}
Será usado a versão 4.4.10 do DSL, com o Kernel 2.4.31, em um placa de modelo PCM-9375, munida de um processador AMD Geode™ LX80 e suporte para comunicação USB, RJ-45, RS-232 e saídas de vídeo CRT e TFT LCD. É de opção do aluno apresentar também o experimento em seu computador pessoal ou em algum dos notebooks fornecidos pela universidade na sala.
%\hfill \today
	
\begin{itemize}
	\item PCM-9375
	\item Damn Small Linux (v4.4.10)
	\item Distribuição Linux qualquer para remasterização do DSL
	\item GCC
\end{itemize}	
	





	
	\section{Desenvolvimento}\label{hardsoft}
		% Deve resumir o experimento, os resultados e a discussão (comentários)
% e incluir opiniões do grupo e propostas futuras para o experimento
%Descrição do hardware
%Descrição do software
%\cite{Advpdf}
%\cite{thepdf}
%\cite{dsl}
Na seção desenvolvimento deve ser respondidas as seguintes perguntas:
 
	\begin{itemize}
		\item Estrutura do Programa: Qual a estruturação/arquitetura do Programa?
	 	\item Qual é o procedimento para a execução do programa? 
	 	\item Quais artefatos são necessários para a execução do programa?
	 	\item Quais os problemas técnicos enfrentados no desenvolvimento do programa?
	 	\item Como os pontos relacionados à disciplina foram abordados no problema? Quais as lições aprendidas? Quais as principais dificuldades?
	 	\item Quais elementos teóricos abordado na disciplina foram implementados no programa?
	 	\item Quais adaptações, extensões, bibliotecas externas, foram necessários para a solução do problema?
	 	\item Caso use parte de códigos disponibilizados na Web, colocar referência \footnote{A home-page de onde tirei
este material:\url{http://en.wikibooks.org/wiki/LaTeX}.Estou formatando para \LaTeX apenas para os estudantes irem se orientando de como e o quê escrever.Assim, me isento de responsabilidade sobre o conteúdo deste texto. Dúvidas: carla(rocha.carla@gmail.com)}
	\end{itemize}
	
	As Figuras são simplesmente inseridas como mostrado na Fig. \ref{Fig1}
	
\begin{figure}[ht]
  \centering
  %\includegraphics[width=5cm]{figs/samplefigure}
   \caption{Arquitetura do Programa.}
  \label{Fig1}
\end{figure}
 
\subsection{Artefatos}
\label{SebSec:Artefatos}
Os artefatos entregues devem ser documentados no relatório:
\begin{itemize}
\item Arquivos contidos no programa. Lista dos nomes dos arquivos, assim como a extensão dos arquivo
\item Aquivo README, com instruções de uso do software desenvolvido e necessidades técnicas para a execução do programa
\item Arquivos de entrada/saída, caso necessário.
\end{itemize}


	
	\section{Caso de Teste}\label{resultados}
		%Deve conter a tabela verdade obtida em laboratorio comentada
%(principalmente nos estados ambíguos)
%Deve apresentar também as dificuldades encontradas no experimento

%Apresentação dos resultados referentes à confecção do modelo em estudo e medidas realizadas, 
%\begin{SCfigure}[1][h] %%Cuidado..ela não se mantem no lugar
%  \centering
%  \includegraphics[width=4cm]{./fts/unb}
%  \caption{Legenda da figura \ref{caption1}. Coloca qualquer coisa.}
%  \label{caption1}
%\end{SCfigure}
%em forma de texto, tabelas e gráficos, como visto na figura \ref{caption} ou na figura \ref{caption1}. Também pode ser incluida nos anexos, como a figura \ref{unb}.







	\ifCLASSOPTIONcaptionsoff
	  \newpage
	\fi

	\bibliographystyle{IEEEtran}%{abnt-num}%{ieeetr}%{abnt-alf}
	\bibliography{bibliography}		% expects file "bibliography.bib" 

	%\else
	%\begin{thebibliography}{6} 
	%O Número implica na quantidade MÁXIMA de itens que pode haver de bibliografias.
	%Altere-o se tiver usado mais que duas fontes.

	%\bibitem{IEEEhowto:kopka}
	%H.~Kopka and P.~W. Daly, \emph{A Guide to \LaTeX}, 3rd~ed.\hskip 1em plus
	%  0.5em minus 0.4em\relax Harlow, England: Addison-Wesley, 1999.

	%\bibitem{Wakerly}
	%John F. Wakerly, \emph{Digital Design: Principles and Practices}, 3rd~ed.
	%	\hskip 1em plus 0.5em minus 0.4em\relax Prentice Hall, 1999.

	%\bibitem{zelenovsky}
	%Mendonça A. e Zelenovsky R., \emph{Eletrônica Digital: Curso Prático e Exercícios}, MZ Editora, Brasil, 2004.
	%Contemporary Logic Design - Katz R. H., First Edition, 1993.	

	%\end{thebibliography}
	%\fi

	\begin{IEEEbiography}[{\includegraphics[width=1in,height=1.25in,clip,keepaspectratio]{./fts/luiz}}]{\luiz}\label{luizbio}
	
	Matricula: \luizmatricula
	
	E-mail: \eluiz
	\end{IEEEbiography}
	
	\begin{IEEEbiography}{\gust}\label{gustbio}
	
	Matricula: \gustmatricula
	
	E-mail: \egust
	\end{IEEEbiography}
	
	\begin{IEEEbiography}{\fay}\label{faybio}
	
	Matricula: \faymatricula
	
	E-mail: \efay
	\end{IEEEbiography}

%\newpage
\onecolumn
\appendices
		\section{Códigos Fontes}\label{src}

%------------------------------------------------------------------------------%		
% Anexar source main.c com link
%\hypertarget{main}{
%\lstinputlisting[language=C]{../main.c} }
%anexar source main.c sem link
%\lstinputlisting[language=C]{../main.c}
%------------------------------------------------------------------------------%		
%\cite{ti_exemplos}
%\tableofcontents %Índice de conteúdos
%\listoftables %Lista de tabelas
%\listoffigures %Lista de figuras
%------------------------------------------------------------------------------%		

\lstinputlisting[language=C]{../../trunk/camera.h}
\lstinputlisting[language=C]{../../trunk/entidade.h}
\lstinputlisting[language=C]{../../trunk/framerate.h}
\lstinputlisting[language=C]{../../trunk/map.h}
\lstinputlisting[language=C]{../../trunk/textureloader.h}
\lstinputlisting[language=C]{../../trunk/defines.h}
\lstinputlisting[language=C]{../../trunk/eventos.h}
\lstinputlisting[language=C]{../../trunk/text.h}


\lstinputlisting[language=C]{../../trunk/camera.cpp}
\lstinputlisting[language=C]{../../trunk/entidade.cpp}
\lstinputlisting[language=C]{../../trunk/framerate.cpp}
\lstinputlisting[language=C]{../../trunk/map.cpp}
\lstinputlisting[language=C]{../../trunk/textureloader.cpp}
\lstinputlisting[language=C]{../../trunk/defines.cpp}
\lstinputlisting[language=C]{../../trunk/eventos.cpp}
\lstinputlisting[language=C]{../../trunk/gamemanager.cpp}
\lstinputlisting[language=C]{../../trunk/text.cpp}
\lstinputlisting[language=C]{../../trunk/tile.cpp}

\onecolumn
\section{Anexos}\label{anexo}

%Inclusão de Figuras
%--------------------Figura logo da UnB----------------------------------------%
%\begin{figure}[h]
%	\centering
%	\includegraphics [scale=1,angle=0,keepaspectratio=true]{./fts/unb}
%	\caption{Logo da UnB}
%	\label{unb}
%\end{figure}
%--------------------Figura logo da UnB----------------------------------------%
%\begin{SCfigure}[1][h] %%Cuidado..ela não se mantem no lugar
%  \centering
%  \includegraphics[width=4cm]{./fts/unb}
%  \caption{Logo da UnB.}
%  \label{unb1}
%\end{SCfigure}


%\begin{multicols}{4}
\tikzstyle{cloud} = [draw, ellipse,fill=red!20, node distance=3cm,
    minimum height=2em]
\tikzstyle{phanton} = []   
\tikzstyle{line} = [->,bend left] %[draw, -latex']
\tikzstyle{arrow} = [loop above]

\begin{center}
\begin{tikzpicture}[node distance = 2cm]
	\tiny\ttfamily
	%-- Estados
	\node [cloud] (E0) at(0,2) {E0};
	\node [cloud] (E1) at(2,1) {E1};
	\node [cloud] (E2) at(2,-1) {E2};
	\node [cloud] (E3) at(0,-2) {E3};
	\node [cloud] (E4) at(-2,-1) {E4};
	\node [cloud] (E5) at(-2,1) {E5};
	%-- setas
	\path (E0) edge [line] (E1);
	\path (E1) edge [line] (E2);
	\path (E2) edge [line] (E3);
	\path (E3) edge [line] (E4);
	\path (E4) edge [line] (E5);
	\path (E5) edge [line] (E0);
	%-- Desvios
	\path (E0) edge [draw,loop above] node{controle1=1} (E0);
	\path (E3) edge [draw,loop below] node{controle2=1} (E3);
	\path (E2) edge [bend right,->]  node[anchor=east]{controle1=1\&\&controle2=0} (E0);
	\path (E5) edge [bend right,->]  node[anchor=west]{controle1=0\&\&controle2=1} (E3);
\end{tikzpicture}
\\\hypertarget{diagrama}{Diagrama de Estados}
\end{center}


%\end{multicols}


		
		
\end{document}


