%Definir, caracterizar e mostrar um breve historico e as utilidades do foco



%Segue um exemplo de como INICIAR o texto. Da segunda palavra em
%diante, escreva normalmente
%	\IEEEPARstart{T}{his} demo file is intended to serve as a ``starter file''
%	for IEEE Computer Society journal papers produced under \LaTeX\ using
%	IEEEtran.cls version 1.7 and later.
%
%	\IEEEPARstart{D}{escrição}, com textos e imagens, da configuração do procedimento e cenário experimental. Explorar técnicas e métodos de medida 
%
%\nocite{dsl} %Apresenta a referência, porém não cita no texto

\IEEEPARstart{D}{escrição} das principais funcionalidades do programa:

\begin{itemize}
	\item Objetivo do programa: Quais são os principais objetivos do presente programa dentro do contexto da disciplina. Listar os principais pontos a serem abordados no programa
	\item Entradas do Programa: como parametrizar tais entradas? Qual o formato dos parâmetros de entrada? Quais são os parâmetros internos do programa?(Como mostrado na tabela \ref{tabela:FuncoesProg})
	\item Saidas Do Programa: Quais são as saídas do programa? Qual o formato dessas saídas?
\end{itemize}

 \begin{table}[h]
\begin{center}
\caption{Principais Funções Implementadas.}
\label{tabela:FuncoesProg}
\begin{tabular}{l p{5cm}}  %define a quantidade de colunas na tabela
Função & Descrição\\ \hline\hline
\textbf{getTitle(self,url)} & Função que recebe uma url de um pdf de um capítulo do redbook e imprime o título   \\ \hline
\textbf{indexToString(self)} & Função que retorna uma string com todos o índice criado  \\ \hline
\textbf{salvar(self,path)} & Função que cria um arquivo com o índice criado  \\ \hline
\textbf{createIndex(self)} & Função que cria o índice  \\ \hline
\textbf{getIndex(self)} & Função que retorna os elementos do índice  \\ \hline
\textbf{printIndex(self)} & Função que imprime os elementos do índice  \\ \hline
\textbf{printUrl(self)} & Função que imrime as url que são usadas pela createIndex()  \\ \hline
\end{tabular}
\end{center}
\end{table}
