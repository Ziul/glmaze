%Definir, caracterizar e mostrar um breve historico e as utilidades do foco



%Segue um exemplo de como INICIAR o texto. Da segunda palavra em
%diante, escreva normalmente
%	\IEEEPARstart{T}{his} demo file is intended to serve as a ``starter file''
%	for IEEE Computer Society journal papers produced under \LaTeX\ using
%	IEEEtran.cls version 1.7 and later.
%
%	\IEEEPARstart{D}{escrição}, com textos e imagens, da configuração do procedimento e cenário experimental. Explorar técnicas e métodos de medida 
%
%\nocite{dsl} %Apresenta a referência, porém não cita no texto

\IEEEPARstart{E}{ste} programa , \assunto, trás não apenas as lições ensinadas em sala de aula, mas também alguns conhecimentos adquiridos no decorrer do curso de engenharia que serão compartilhados neste documento.

\subsection{Objetivos}\label{obj}


No inicio do projeto, tínhamos os seguintes desafios:

\begin{itemize}
	\item Criar um programa que faça de uso das ferramentas do OpenGL.
	\item Aperfeiçoar o conhecimento da linguagem $C$ para viabilizar a construção de um programa com grande volume de dados de forma pratica e passível de modulação.
\end{itemize}

Devido ao OpenGL ser uma ferramenta bastante conhecida, é extremamente fácil encontrar na internet exemplos e modelos utilizando a ferramenta, porém com o decorrer do projeto, o grupo tratou de incluir alguns novos itens como desafios para o projeto, a fim de melhorar a qualidade do produto final. Estes foram os pontos incluídos:

\begin{itemize}
	\item \textbf{Uso da linguagem C++}, no intuito de aproveitar o conceito de orientação de objetos para expandir o projeto para um jogo mais próximo de algo com formato profissional.
	\item \textbf{Caracterização dos módulos}, dividindo assim o programa em vários arquivos fontes menores, facilitando assim a localização de $bugs$ e permitindo também a possibilidade de que varias pessoas editem o código simultaneamente.
	\item \textbf{Uso de ferramentas VCS/SVN}, permitindo vários backups e facilitando a construção de varias partes do código em múltiplos computadores.
	\item \textbf{Portabilidade}. O conhecimento de que o OpenGL não se restringia apenas a plataforma $Windows$ acabou gerando o desejo de produzir um código que pudesse ser compilado em qualquer computador, seja $Windows$, $Mac$ ou $Linux$.
\end{itemize}

\end{multicols}
\begin{figure}[h]
\begin{center}
  \centering
  \includegraphics[scale=0.45]{./fts/lvl1}
  \caption{Pac-Man.\\O clássico dos anos 80 só foi ter um score perfeito - máximo de pontos, sem falhas ou mortes - em 1999, quando \textit{Billy Mitchell} consegui a incrível marca de $3,333,360$ pontos, após vencer os consecutivos 256 leveis do jogo.}
  \label{pac1}
\end{center}
\end{figure}
\begin{multicols}{2}

\input game

\subsection{Entradas e Saídas}\label{inputs}

Inicialmente, o grupo precisava de uma sala complexa, com varias paredes e corredores. Assim poderíamos levantar estruturas de colisões, movimentação, iluminação e texturas. De inicio, foi utilizado um algoritmo chamado e \textit{"Growing Tree"}, utilizado para a criação de labirintos. Inicialmente foram escolhidos dois programas base para a criação de um labirinto randômico e posteriormente a exportação do labirinto para o programa.

Com a evolução do programa e as ferramentas feitas, foi adotado um labirinto fixo, que tivesse as características dos jogos clássicos de PAC-MAN, que pode ser observado na figura \ref{pac1}.

%\end{multicols}
%\begin{SCfigure}[5][h] %%Cuidado..ela não se mantem no lugar
%	\centering
%	\includegraphics[scale=0.3]{./fts/lvl1}
%	\caption{Pac-Man.\\O clássico dos anos 80 só foi ter um score perfeito - máximo de pontos, sem falhas ou mortes - em 1999, quando \textit{Billy Mitchell} consegui a incrível marca de $3,333,360$ pontos, após vencer os consecutivos 256 leveis do jogo.}
%	\label{pac1}
%\end{SCfigure}
%\begin{multicols}{2}

O programa ainda continua fazendo leituras do teclado e do mouse para a movimentação do usuário, apresentando apenas como saída o \textit{framebuffer} na tela do usuário.



%--------------------------------------------------------------------------

%\begin{itemize}
%	\item Objetivo do programa: Quais são os principais objetivos do presente programa dentro do contexto da disciplina. Listar os principais pontos a serem abordados no programa
%	\item Entradas do Programa: como parametrizar tais entradas? Qual o formato dos parâmetros de entrada? Quais são os parâmetros internos do programa?(Como mostrado na tabela \ref%{tabela:FuncoesProg})
%	\item Saidas Do Programa: Quais são as saídas do programa? Qual o formato dessas saídas?
%\end{itemize}

% \begin{table}[h]
%\begin{center}
%\caption{Principais Funções Implementadas.}
%\label{tabela:FuncoesProg}
%\begin{tabular}{l p{5cm}}  %define a quantidade de colunas na tabela
%Função & Descrição\\ \hline\hline
%\textbf{getTitle(self,url)} & Função que recebe uma url de um pdf de um capítulo do redbook e imprime o título   \\ \hline
%\textbf{indexToString(self)} & Função que retorna uma string com todos o índice criado  \\ \hline
%\textbf{salvar(self,path)} & Função que cria um arquivo com o índice criado  \\ \hline
%\textbf{createIndex(self)} & Função que cria o índice  \\ \hline
%\textbf{getIndex(self)} & Função que retorna os elementos do índice  \\ \hline
%\textbf{printIndex(self)} & Função que imprime os elementos do índice  \\ \hline
%\textbf{printUrl(self)} & Função que imrime as url que são usadas pela createIndex()  \\ \hline
%\end{tabular}
%\end{center}
%\end{table}
