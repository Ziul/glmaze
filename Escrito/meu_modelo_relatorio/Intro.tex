%Definir, caracterizar e mostrar um breve historico e as utilidades do foco



%Segue um exemplo de como INICIAR o texto. Da segunda palavra em
%diante, escreva normalmente
%	\IEEEPARstart{T}{his} demo file is intended to serve as a ``starter file''
%	for IEEE Computer Society journal papers produced under \LaTeX\ using
%	IEEEtran.cls version 1.7 and later.
%
%	\IEEEPARstart{D}{escrição}, com textos e imagens, da configuração do procedimento e cenário experimental. Explorar técnicas e métodos de medida 
%
\nocite{dsl} %Apresenta a referência, porém não cita no texto
\subsection{Objetivos}\label{obj}

\IEEEPARstart{E}{ste} experimento tem como objetivo descrever um passo a passo de como obter um ambiente básico para as demais aulas. Para isso, precisamos ter uma distribuição contendo os aplicativos fundamentais para o desenvolvimento da matéria de Sistemas Embarcados.
	
\subsection{Introdução Teórica}\label{intro}	
\subsubsection{Damn Small Linux}\label{dsl}
\textit{Damn Small Linux}, ou apenas DSL, é uma distribuição versátil do Linux, tendo o tamanho de cerca de 50mb e ocupando como espaço final na RAM cerca de 128MB. A distribuição foi montada com o intuito de cumprir os seguintes objetivos:
	
%\hfill \today

\subsubsection{Material}\label{mat}
Será usado a versão 4.4.10 do DSL, com o Kernel 2.4.31, em um placa de modelo PCM-9375, munida de um processador AMD Geode™ LX80 e suporte para comunicação USB, RJ-45, RS-232 e saídas de vídeo CRT e TFT LCD. É de opção do aluno apresentar também o experimento em seu computador pessoal ou em algum dos notebooks fornecidos pela universidade na sala.
%\hfill \today
	
\begin{itemize}
	\item PCM-9375
	\item Damn Small Linux (v4.4.10)
	\item Distribuição Linux qualquer para remasterização do DSL
	\item GCC
\end{itemize}	
	




