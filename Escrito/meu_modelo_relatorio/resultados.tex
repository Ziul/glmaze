%Deve conter a tabela verdade obtida em laboratorio comentada
%(principalmente nos estados ambíguos)
%Deve apresentar também as dificuldades encontradas no experimento

%Apresentação dos resultados referentes à confecção do modelo em estudo e medidas realizadas, 
%\begin{SCfigure}[1][h] %%Cuidado..ela não se mantem no lugar
%  \centering
%  \includegraphics[width=4cm]{./fts/unb}
%  \caption{Legenda da figura \ref{caption1}. Coloca qualquer coisa.}
%  \label{caption1}
%\end{SCfigure}
%em forma de texto, tabelas e gráficos, como visto na figura \ref{caption} ou na figura \ref{caption1}. Também pode ser incluida nos anexos, como a figura \ref{unb}.
%\lipsum[2]

Nessa seção deve ser apresentado pelo menos um exemplo de caso de teste. Se não for especificado na descrição do problema, ela deve definida, explicada e ilustrada pelos autores.

\section{Conclusão}\label{conclusion}

Discutir os principais pontos relativos ao desenvolvimento do programa:

\begin{itemize}
	\item Dificuldades encontradas em atingir os objetivos propostos. Caso não tenha sido possível, concluir 100\% da tarefa, listar razões para tal.
	\item Sugestões de melhorias do programa.
	\item Pontos teóricos mais relevantes abordados na prática e a relevância de tais conceitos (Exemplo de aplicações que tais conceitos seriam úteis). Com citações  se necessário.
\end{itemize}


