%Deve conter a tabela verdade obtida em laboratorio comentada
%(principalmente nos estados ambíguos)
%Deve apresentar também as dificuldades encontradas no experimento

%Apresentação dos resultados referentes à confecção do modelo em estudo e medidas realizadas, 
%\begin{SCfigure}[1][h] %%Cuidado..ela não se mantem no lugar
%  \centering
%  \includegraphics[width=4cm]{./fts/unb}
%  \caption{Legenda da figura \ref{caption1}. Coloca qualquer coisa.}
%  \label{caption1}
%\end{SCfigure}
%em forma de texto, tabelas e gráficos, como visto na figura \ref{caption} ou na figura \ref{caption1}. Também pode ser incluida nos anexos, como a figura \ref{unb}.





