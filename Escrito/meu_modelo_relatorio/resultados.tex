%Deve conter a tabela verdade obtida em laboratorio comentada
%(principalmente nos estados ambíguos)
%Deve apresentar também as dificuldades encontradas no experimento

%Apresentação dos resultados referentes à confecção do modelo em estudo e medidas realizadas, 
%\begin{SCfigure}[1][h] %%Cuidado..ela não se mantem no lugar
%  \centering
%  \includegraphics[width=4cm]{./fts/unb}
%  \caption{Legenda da figura \ref{caption1}. Coloca qualquer coisa.}
%  \label{caption1}
%\end{SCfigure}
%em forma de texto, tabelas e gráficos, como visto na figura \ref{caption} ou na figura \ref{caption1}. Também pode ser incluida nos anexos, como a figura \ref{unb}.
%\lipsum[2]

\subsection{Dificuldades encontradas}\label{dificuldades}

\begin{itemize}
	\item Dificuldades em discubrir o modo com que o glut atrubui as funções e gerencia os eventos.
	\item Dificuldades em tornar o jogo jogável por multiplataformas; especificamente no tratamento de sons.
	\item Dificuldade em imprimir objetos 2d por cima do cenario 3d (minimap)
\end{itemize}

\subsection{Sugestões}\label{sugest}

\begin{itemize}
	\item Multiplayer para 2 jogadores Alternados.
	\item Registro de nome para usuarios que concluirem um nível com sua respectiva potuação.
\end{itemize}

%\subsubsection{Pontos relevantes abortados}\label{relevantes} %que seriam uteis na prática e a relevância de tais conceitos (Exemplo de aplicações que tais conceitos seriam úteis). Com citações  se necessário.


