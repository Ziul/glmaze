\subsection{A Historia}\label{history}


Toru Iwatani, criador do jogo PAC-MAN, se inspirou em uma história infantil sobre uma criatura que protegia as crianças dos monstros por comê-los. Um dos métodos de design Iwatani incluído a palavras-chave associadas com uma história para auxiliar no desenvolvimento de suas idéias. O kanji da palavra taberu ("comer"), tornou-se a premissa para o jogo. A palavra kuchi ("boca") tem um formato quadrado para seu símbolo kanji e forneceu a inspiração para o jogo da principal lenda personagem-o mais conhecido de Iwatani receber sua inspiração de uma pizza com uma fatia faltando foi, por sua própria admissão, não inteiramente correta: 

\begin{quote}
	\textit{"Bem, é uma meia verdade. Em caráter do japonês para boca (Kuchi) tem uma forma quadrada. Não é circular como a pizza, mas eu decidi arredonda-lo. Havia a tentação de fazer a forma de Pac-Man menos simples. Enquanto eu estava projetando este jogo, alguém sugeriu adicionar os olhos. Mas nós finalmente descartamos essa idéia, porque uma vez que nós adicionacemos olhos, nós gostariamos de adicionar óculos e talvez um bigode. Não teria fim. O alimento é a outra parte do conceito básico. Na minha concepção inicial, eu tinha colocado o jogador em meio a comida por toda a tela. Entao eu pensei sobre isso, percebi que o jogador não saberia exatamente o que fazer: o objetivo do jogo seria obscuro. Então, eu criei um labirinto e coloquei a comida nele. Assim, quem jogasse o jogo teria alguma estrutura ao se mover através do labirinto. Os japoneses têm uma gíria - paku-paku - eles usam para descrever o movimento da boca abrindo e fechando, enquanto se come. O nome Puck-Man veio essa palavra. "}

- Toru Iwatani
\end{quote}

Os monstros da história das crianças foram incluídos como quatro fantasmas que perseguem o jogador através do labirinto, proporcionando um elemento de tensão. Ataques contra o jogador foram projetados para vir em ondas (semelhante ao \textbf{Space Invaders}), em oposição a um ataque sem fim, e cada fantasma foi dada uma personalidade única e caráter. A história das crianças também incluiu o conceito de kokoro ("espírito") ou uma força de vida utilizada pela criatura que lhe permitia comer os monstros. Toru incorporou este aspecto da história de quatro pastilhas de energia comestíveis no labirinto para virar a mesa contra os fantasmas, tornando-os vulneráveis ​​a ser comido pelo jogador.

A aparência de Puck-Man continuou a evoluir por mais de um ano. Uma grande quantidade de tempo e esforço foi feito para desenvolver os fantasmas padrões de movimentos únicos através do labirinto e aprimorando as variáveis ​​do jogo de dificuldade, como placas foram apuradas. Símbolos de bônus (incluindo o carro-chefe Galaxian) foram adicionados à mistura, em algum momento, e os fantasmas foram finalmente nomeados: Akabei, Pinky, Aosuke, e Guzuta. Efeitos sonoros e música foram alguns dos toques finais adicionados, com o desenvolvimento se aproximando do fim, eram feitos ajustes constante do comportamento dos fantasmas.

Midway era uma distribuidora de jogos que funcionam com moedas nos EUA. Estavam sempre procurando o próximo grande sucesso do Japão para licenciar e trazer para a América. Eles optaram por tanto Puck-Man e Galaxian, modificando os armários e obras de arte para torná-los mais fáceis de fabricar, bem como proporcionar um olhar mais americano.

Puck-Man passou por grandes mudanças: o gabinete foi ligeiramente modificado, mudando a cor de branco para um amarelo brilhante para fazê-lo sobressair no arcade. O detalhado gabinete multi-colorido foi substituído com mais barato, para produzir em três cores de arte que ilustra uma representação icônica de Puck-Man (agora desenhado com olhos e pés) e um fantasma azul. Nomes ingleses foram dadas para os fantasmas (Blinky, Pinky, Inky e Clyde), e o título foi mudado da Namco para  a Midway. A mudança mais significativa para Puck-Man foi o nome. A Midway temia que seria muito fácil para vândalos desagradável de espírito para mudar o P em Puck-Man para um F, criando um epíteto desagradável. Não querendo seu produto associado a esta palavra, a Midway renomeou o jogo para Pac-Man antes de liberá-lo para os arcades americanos em outubro de 1980.\cite{dossier}
